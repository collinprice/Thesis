\begin{table}
\caption{Minimum Move Required at 1\%}
\label{fig:localParameters}
\centering
\begin{tabular}{ | l | l | l | l | l | }
  \hline
    Element & Min. \\ \hline
    O & 0.025 \\ \hline
    Mn & 0.01 \\ \hline
    Ca & 1 \\ \hline
    C & 0.5 \\ \hline
    N & 0.5 \\ \hline
    H & 5 \\ \hline
\end{tabular}
\end{table}

\begin{table}
\caption{Minimum Move Required at 5\%}
\label{fig:localParameters}
\centering
\begin{tabular}{ | l | l | l | l | l | }
  \hline
    Element & Min. \\ \hline
    O & 0.5 \\ \hline
    Mn & 0.5 \\ \hline
    Ca & 5 \\ \hline
    C & 5 \\ \hline
    N & 5 \\ \hline
    H & 5 \\ \hline
\end{tabular}
\end{table}

We performed some analysis on OEC in order to learn more about how moving individual atoms would affect the RMSD score of the EXAFS. We moved each atom, individually, in a variety of directions and calculated its RMSD score. Each atom was moved in a total of six directions; $\pm$X, $\pm$Y, and $\pm$Z. At a variety of distances; 0.001$\AA$, 0.005$\AA$, 0.01$\AA$, 0.025$\AA$, 0.05$\AA$, 0.1$\AA$, 0.5$\AA$, 1$\AA$, and 5$\AA$. This was done to determine how much movement was required of an atom to make a significant change to the RMSD score.

The value of 0.05$\AA$ was chosen for our experiments as a middle ground that could be applied to each of our chemical elements. It should be noted that the value of 0.05$\AA$ is particular to OEC. A similar analysis could be done to determine the minimum move distance for each element in another chemical complex.