\section{Genetic Algorithm: Post-Optimization}

\subsection{Purpose}

- answers were close so we tried local search technique
- suggested based on previous work maybe?

\subsection{Setup}

- how populations were generated
- representation differences
- explain Initial Movement Radius
- fitness function

\subsection{System Parameters}

\begin{table}
	\label{table:post-op-pso}
	\centering
	\begin{tabular}{ | >{\bfseries}c | c | c | c | }
		\hline
		Experiment Set & 1 & 2 & 3 \\ \hline
		Initial Movement Radius & 0.05 & 0.05 & 0.25 \\ \hline
		Generations & 100 & 30 & 30 \\ \hline
		Population Size & 50 & 50 & 50 \\ \hline
	\end{tabular}
	\caption{System parameters for the Post-Optimization PSO runs}
\end{table}

\begin{table}
	\label{table:post-op-de}
	\centering
	\begin{tabular}{ | >{\bfseries}c | c | c | }
		\hline
		Experiment Set & 1 & 2 \\ \hline
		Initial Movement Radius & 0.05 & 0.25 \\ \hline
		Generations & 30 & 30 \\ \hline
		Population Size & 50 & 50 \\ \hline
	\end{tabular}
	\caption{System parameters for the Post-Optimization DE runs}
\end{table}

\subsection{Analysis}

- reasons DE might not have worked
- benefits of post op