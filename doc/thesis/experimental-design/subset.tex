\section{Atom Subsets}

\subsection{Purpose}

The purpose of this study is to attempt to reduce the search space of the structure refinement problem. Table~\ref{table:atom-breakdown} outlines the number of atoms within the OEC. If we could reduce the number of atoms needed to move during the evolutionary process it would reduce the search space. This study will show how a GA performs on the structure refinement problem when certain chemical elements are kept rigid. A rigid chemical element means that the chemical element will not be evolved during the run. The position of the rigid chemical elements will be the same in each individual in the population. 

\begin{table}
	\centering
	\begin{tabular}{ | l | l | }
		\hline
		\textbf{Chemical Element} & \textbf{Sum} \\ \hline
		Mn & 4 \\ \hline
		Ca & 1 \\ \hline
		O & 26 \\ \hline
		C & 14 \\ \hline
		N & 6 \\ \hline
		H & 28 \\ \hline
	\end{tabular}
	\caption{Chemical Element Breakdown}
	\label{table:atom-breakdown}
\end{table}

\subsection{System Parameters}

A basic GA was used in this experiment. The system parameters used during the experiment are shown in Table~\ref{table:subset-parameters}. Table~\ref{table:subset-setup} outlines the different experiments that were run. Each experiment contains a different combination of rigid and flexible chemical elements but each experiment used the same GA system parameters. The fitness function used for GA is defined in Subsection~\ref{sec:fitness-exafs}.

\begin{table}
	\centering
	\begin{tabular}{ l r }
		\hline
		Runs & 10 \\
		Population size & 50 \\
		Crossover rate & 0.7 \\
		Mutation rate & 0.3 \\
		Elitism & True \\
		\hline
	\end{tabular}
	\caption{GA Subset Parameters}
	\label{table:subset-parameters}
\end{table}

\begin{table}
	\centering
	\begin{tabular}{ | >{\bfseries}p{2cm} | p{1cm} | p{1cm} | p{1cm} | p{1cm} | p{1cm} | }
		\hline
		Exp. Set & 1 & 2 & 3 & 4 & 5 \\ \hline
		Flexible Atoms & Mn, Ca, C, O, N, H & Mn, Ca, C, O, N & Mn, Ca, C & Mn, Ca, O & Mn, Ca, N \\ \hline
		Rigid Atoms &  & H & H, N, O & H, N, C & H, C, O \\ \hline
	\end{tabular}
	\\
	\vspace{3 mm}
	\begin{tabular}{ | >{\bfseries}p{2cm} | p{1cm} | p{1cm} | p{1cm} | p{1cm} | }
		\hline
		Exp. Set & 6 & 7 & 8 & 9 \\ \hline
		Flexible Atoms & Mn, Ca, C, O & Mn, Ca, C, N & Mn, Ca, O, N & Mn, Ca \\ \hline
		Rigid Atoms & H, N & H, O & H, C & H, C, O, N \\ \hline
	\end{tabular}
	\caption{Experiments with different subsets}
	\label{table:subset-setup}
\end{table}