\section{Genetic Algorithm: Viability}

\subsection{Purpose}

The purpose of this study is to determine the viabiltiy of using an evolutionary algorithm to solve the structure refinement problem (Section~\ref{sec:problem-definition}). Previous studies~\cite{sproviero2008model}~\cite{luber2011s1} have shown that iterative algorithms work well in finding candidate solutions to the problem.

\subsection{Setup}

During the initial stages of experimentation the initial populations were created by randomly adjusting the atoms within this initial structure. To create a new individual each atom within the atomic structure would be randomly moved by 0.05-0.5\AA. This form of population generation was quickly discarded because many of these individuals were either chemically infeasible or generated erroneous EXAFS spectra. 

- what population type was used.
- representation
- fitness function

\subsection{System Parameters}

Two evolutionary algorithms were used in this run: GA, and RGA. The system parameters for the GA can be seen in Table~\ref{table:viability-ga}, and the system parameters for the RGA can be seen in Table~\ref{table:viability-rga}.

- mutation operator
- selection
- crossover

\begin{table}
	\label{table:viability-ga}
	\centering
	\begin{tabular}{ | >{\bfseries}c | c | c | c | c | c | c | }
		\hline
		Experiment Set & 1 & 2 & 3 & 4 & 5 & 6 \\ \hline
		Crossover Rate & 80 & 80 & 70 & 80 & 70 & 80 \\ \hline
		Mutation Rate & 20 & 10 & 30 & 10 & 30 & 20 \\ \hline
		Elitism & False & False & False & True & True & True \\ \hline
		Generations & 30 & 30 & 30 & 30 & 30 & 30 \\ \hline
		Population Size & 50 & 50 & 50 & 50 & 50 & 50 \\ \hline
		Tourament Size & 3 & 3 & 3 & 3 & 3 & 3 \\ \hline
	\end{tabular}
	\caption{System parameters for the basic GA runs}
\end{table}

\begin{table}
	\label{table:viability-rga}
	\centering
	\begin{tabular}{ | >{\bfseries}c | c | c | c | c | c | c | }
		\hline
		Experiment Set & 1 & 2 & 3 & 4 & 5 & 6 \\ \hline
		Crossover Rate & 80 & 80 & 80 & 80 & 70 & 70 \\ \hline
		Mutation Rate & 20 & 20 & 20 & 20 & 30 & 30 \\ \hline
		Elitism & True & True & True & True & True & True \\ \hline
		Generations & 67 & 81 & 107 & 124 & 67 & 81 \\ \hline
		Population Size & 50 & 50 & 50 & 50 & 50 & 50 \\ \hline
		Tourament Size & 3 & 3 & 3 & 3 & 3 & 3 \\ \hline
		Convergence Rate & 10 & 5 & 10 & 5 & 5 & 10 \\ \hline
		Number of Restarts & 3 & 3 & 5 & 5 & 5 & 5 \\ \hline
	\end{tabular}
	\caption{System parameters for the RGA runs}
\end{table}

\subsection{Analysis}

- numbers of generations
- rga sees more individuals
- examples of rga