\section{Alternative Algorithms}

\subsection{Purpose}

The purpose of this study is to determine how well DE, and PSO perform on the structure refinement problem. In Section~\ref{sec:ga-post-op} DE, and PSO were used as a post-optimization of the results found in Section~\ref{sec:ga-experiment}. In this study DE, and PSO will be more directly compared with the algorithms used in Section~\ref{sec:ga-experiment}.

\subsection{Population}

The initial population for both the DE, and PSO will be created from a random sampling of the 300 candidate solutions that were generated using the molecular dynamics simulation discussed in Subsection~\ref{subsec:molecular-population}.

\subsection{System Parameters}

Two algorithms were used in this experiment: DE, and PSO. The system parameters for the DE can be seen in Table~\ref{table:ea-de}, and the system parameters for the PSO can be seen in Table~\ref{table:ea-pso}. The fitness function used for the DE, and PSO is defined in Subsection~\ref{sec:fitness-exafs}.

Representation 1 (See Subsection~\ref{subsec:encoding-1}) was chosen as the individual representation for both the DE, and PSO. Since each index within the individual is a 3-dimensional coordinate, these values are updated using standard vector arithmetic at each index.

The \textit{velocity parameter} found in Table~\ref{table:ea-pso} represents the random range that was used to generate the initial \textit{velocity vector} of each particle in the PSO. For example, a \textit{velocity parameter} of 0.01 means that each velocity vector was generated using random values between -0.01 and 0.01.

The DE, and PSO experiments were initially run using a variety of different tuning parameters. The PSO algorithm has three tunable user defined parameters: \textit{inertia}, \textit{social} and \textit{cognitive}. Based on the work performed by Eberhart, and Shi~\cite{eberhart2000comparing}, the \textit{social} and \textit{cognitive} parameters were set to 1.496180 and \textit{inertia} was set to 0.729844. The DE algorithm has two tunable user defined parameters: \textit{differential weight}, and \textit{crossover probability}. Several different combinations of values for \textit{differential weight}, and \textit{crossover probability} were used but the parameters that produced the best results were from a research paper by Hvass Laboratories~\cite{pedersen2010good}. \textit{Differential weight} was set to 0.4717, and \textit{crossover probability} was set to 0.8803.

\begin{table}
	\centering
	\begin{tabular}{ | >{\bfseries}c | c | c | c | c | c | c | }
		\hline
		Exp. Set & 1 & 2 & 3 & 4 & 5 & 6 \\ \hline
		Pop. Size & 50 & 50 & 50 & 50 & 50 & 50 \\ \hline
		Gens. & 100 & 100 & 100 & 200 & 200 & 200 \\ \hline
		Velocity & 0.01 & 0.05 & 0.1 & 0.01 & 0.05 & 0.1 \\ \hline
	\end{tabular}
	\\
	\vspace{3 mm}
	\begin{tabular}{ | >{\bfseries}c | c | c | c | c | c | c | }
		\hline
		Exp. Set & 7 & 8 & 9 & 10 & 11 & 12 \\ \hline
		Pop. Size & 100 & 100 & 100 & 100 & 100 & 100 \\ \hline
		Gens. & 100 & 100 & 100 & 200 & 200 & 200 \\ \hline
		Velocity & 0.01 & 0.05 & 0.1 & 0.01 & 0.05 & 0.1 \\ \hline
	\end{tabular}
	\caption{System parameters for the PSO runs}
	\label{table:ea-pso}
\end{table}

\begin{table}
	\centering
	\begin{tabular}{ | >{\bfseries}c | c | c | c | c | }
		\hline
		Experiment Set & 1 & 2 & 3 & 4 \\ \hline
		Population Size & 50 & 50 & 100 & 100 \\ \hline
		Generations & 100 & 200 & 100 & 200 \\ \hline
	\end{tabular}
	\caption{System parameters for the DE runs}
	\label{table:ea-de}
\end{table}

