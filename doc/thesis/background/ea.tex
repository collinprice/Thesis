\section{Evolutionary Algorithms}

An evolutionary algorithm (EA) is a population-based metaheuristic optimization algorithm. An evolutionary algorithm is a search heuristic that is based on Darwin's theory of natural evolution. Darwin theorized that over a period of time a population of individuals would naturally mate and create offspring that were better than themselves. He suggests that not all individuals are created equally and that eventually the weaker individuals would die off. This same principle can be applied to a search algorithm as a heuristic. An EA contains a population of individuals that are evolved to find improved candidate solutions.

Figure~\ref{fig:evolutionaryFlowchart} demonstrates how the basic EA operates. An EA consists of many parts. The following subsections wil explain each of these parts.

\begin{figure}[H]
	\centering
	\includegraphics[bb=0 0 524 481,scale=0.5]{figures/EA.jpeg}
	\caption{Population Individual Modification}
	\label{fig:evolutionaryFlowchart}
\end{figure}

\subsection{Population}

The population is a key peice to an EA. Each individual in the population represents a possible candidate solution to the problem that is trying to be solved. The representation of the individual is usually unique to the problem.

\subsection{Evaluation Function}

This operator determines the fitness of an individual. Each individual is evaluated and given a fitness score to represent how well the individual performed on the problem. This operation is problem specific and can be very difficult to determine how a problem should be evaluated. The evaluation function is important for differentiating individuals. A poor evaluation function can make each of the individuals appear to be similar when they actually have small key differences. 

\subsection{Stopping Criteria}

Stopping criteria is used to determine when the EA should stop evolving. There are generally three ways a stopping criteria can be reached: a maximum number of iterations is reached, the population has converged on the same solution, or the solution has been found.

\subsection{Evolving the Population}

The evolutionary process of an EA is what is different in each implementation of an EA. Each algorithm has a different interpretation of how the population should be evolved. Evolving the population consists of using the individuals in the population to create a new population. Later in this work the different interpretations will be explained.