\section{Evolutionary Algorithms}

\subsection{Analysis}

After viewing the results from the DE, and PSO experiments in Section~\ref{sec:post-op-analysis} these algorithms were suspected to perform well on the structure refinement problem. Table~\ref{table:ea-de-results} shows a summary from the DE experiments and Table~\ref{table:ea-pso-results} provide a summary of the results from the PSO experiments. Each experiment was run a total of 30 times to obtain the average.

\begin{table}
	\centering
	\begin{tabular}{ | >{\bfseries}c | c | c | c | c | c | c | }
		\hline
		Exp. Set & 1 & 2 & 3 & 4 & 5 & 6 \\ \hline
		% Best RMSD & 0.7735 & 0.6840 & 0.7498 & \textbf{0.6058} & 0.6568 & 0.6429 \\ \hline
		% Average RMSD & 0.9121 & 0.9046 & 0.9090 & 0.7743 & 0.7643 & 0.7731 \\ \hline
		Best RMSD & 0.7735 & 0.6840 & 0.7498 & \textbf{0.6109} & 0.6653 & 0.6621 \\ \hline
		Average Best RMSD & 0.9136 & 0.9004 & 0.9049 & 0.8025 & 0.7907 & 0.7933 \\ \hline
	\end{tabular}
	\\
	\vspace{3 mm}
	\begin{tabular}{ | >{\bfseries}c | c | c | c | c | c | c | }
		\hline
		Exp. Set & 7 & 8 & 9 & 10 & 11 & 12 \\ \hline
		% Best RMSD & 0.7392 & 0.6881 & 0.7306 & 0.6440 & 0.6398 & 0.6504 \\ \hline
		% Average RMSD & 0.8724 & 0.8876 & 0.8836 & 0.7563 & 0.7441 & \textbf{0.7369} \\ \hline
		Best RMSD & 0.7392 & 0.6881 & 0.7306 & 0.6546 & 0.6571 & 0.6688 \\ \hline
		Average Best RMSD & 0.8775 & 0.8856 & 0.8836 & 0.7750 & \textbf{0.7714} & 0.7594 \\ \hline
	\end{tabular}
	\caption{Results for the PSO runs}
	\label{table:ea-pso-results}
\end{table}

\begin{table}
	\centering
	\begin{tabular}{ | >{\bfseries}c | c | c | c | c | }
		\hline
		Experiment Set & 1 & 2 & 3 & 4 \\ \hline
		% Best RMSD & 0.9793 & \textbf{0.9405} & 1.0357 & 0.9646 \\ \hline
		% Average Best RMSD & 1.1055 & \textbf{1.0554} & 1.1456 & 1.0635 \\ \hline
		Best RMSD & 0.9793 & 0.9405 & 1.0357 & 0.9646 \\ \hline
		Average Best RMSD & 1.1120 & 1.0540 & 1.1453 & 1.0624 \\ \hline
	\end{tabular}
	\caption{Results for the DE runs}
	\label{table:ea-de-results}
\end{table}

Comparing the results of the DE, and PSO with the results of the GA, and RGA one would notice that the search algorithms that operated on a continuous space performed better than the discrete algorithms.

The DE, and PSO experiments were initially run using a variety of different system parameters. The results shown in Table~\ref{table:ea-de-results} and Table~\ref{table:ea-pso-results} used the following tuning parameters. The PSO algorithm has three tunable user defined parameters: inertia, social and cognitive. Based on the work performed in the research paper~\cite{eberhart2000comparing} the parameters social and cognitive was set to 1.496180, while inertia was set to 0.729844. The DE algorithm has two tunable user defined parameters: differential weight, and crossover probability. Several different combinations of values for differential weight, and crossover probability were used but the parameters that produced the best results were from a research paper from Hvass Laboratories~\cite{pedersen2010good} where differential weight was set to 0.4717, and crossover probability was set to 0.8803.

\begin{figure*}
	\centering
	\begin{tikzpicture}
		\begin{axis}[
			width=14cm,
			height=6cm,
			grid=both,
			title={EXAFS Spectra in k space},
			legend entries={Experimental,Calculated},
			legend pos=south east,
			xlabel={$k \mathbin{/} A\textsuperscript{-1}$},
			ylabel={$EXAFS \chi k\textsuperscript{3}$}
		]

		\addplot[mark=x] table [col sep=comma,y index=1, x index=0] {data/pso-best-exafs-comparison.csv};
		\addplot[mark=*,mark size=1] table [col sep=comma,y index=2, x index=0] {data/pso-best-exafs-comparison.csv};

		\end{axis}
	\end{tikzpicture}
	\caption{OEC EXAFS Spectra Comparison}
	\label{fig:bestRunEXAFS}
\end{figure*}