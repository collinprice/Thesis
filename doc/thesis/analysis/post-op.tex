\section{Genetic Algorithm: Post-Optimization}
\label{sec:post-op-analysis}

\subsection{Analysis}

Table~\ref{table:post-op-de-results} and Table~\ref{table:post-op-pso-results} provide a summary of the results from the DE, and PSO experiments respectively. Each experiment was run a total of 30 times to obtain the average. The two algorithms were performed using individuals generated using two different \textit{initial movement radius}. Using the larger \textit{initial movement radius} of $\pm$0.25\AA\ had a negative impact on the DE, and PSO candidate solutions. The large \textit{initial movement radius} caused the experiments to move from the initial local optima, that was found using the RGA, and move into sub-optimal solutions. As we decreased the \textit{initial movement radius} the results started to improve. Using the smaller \textit{initial movement radius} of $\pm$0.05\AA\ provided the best results.

\begin{table}
	\centering
	\begin{tabular}{ | >{\bfseries}c | c | c | c | }
		\hline
		Experiment Set & 1 & 2 \\ \hline
		% Average Best RMSD & 1.1386 &1.7267  \\ \hline
		Best RMSD & \textbf{0.9973} & 1.4118 \\ \hline
		Average Best RMSD & \textbf{1.1386} & 1.7267 \\ \hline
	\end{tabular}
	\caption{Results of DE Post-Optimization}
	\label{table:post-op-de-results}
\end{table}

\begin{table}
	\centering
	\begin{tabular}{ | >{\bfseries}c | c | c | }
		\hline
		Experiment Set & 1 & 2 \\ \hline
		% Average Best RMSD & 0.9001 & 1.2445 \\ \hline
		Best RMSD & \textbf{0.7977} & 0.9296 \\ \hline
		Average Best RMSD & \textbf{0.9001} & 1.2445 \\ \hline
	\end{tabular}
	\caption{Results of PSO Post-Optimization}
	\label{table:post-op-pso-results}
\end{table}

The results from the DE experiments were never able to improve upon the seed candidate solution. Decreasing the \textit{initial movement radius} improved the results from the DE but it was not enough to local improved candidate solutions.

In contrast, the results from the PSO experiments were very successful. Using the \textit{initial movement radius} of $\pm$0.05\AA\ the PSO was able to greatly improve upon the seed candidate solution. Both the DE, and PSO were run for 30 generations. The DE experiments had reached a convergence but the PSO data was still on a downward slope. The PSO was run for 200 generations and it ended up locating even better solutions. Figure~\ref{fig:post-op-pso-best-exafs} shows the best EXAFS spectra after optimization from the PSO. The RMSD score of the candidate solution was significantly reduced which shows that PSO works very well as a post-optimization strategy. 

\begin{figure*}
	\centering
	\begin{tikzpicture}
		\begin{axis}[
			width=14cm,
			height=6cm,
			grid=both,
			title={EXAFS Spectra in k space},
			legend entries={Experimental,Calculated},
			legend pos=south east,
			xlabel={$k \mathbin{/} A\textsuperscript{-1}$},
			ylabel={$EXAFS \chi k\textsuperscript{3}$}
		]

		\addplot[mark=x] table [col sep=comma,y index=1, x index=0] {data/post-op-pso-exafs-comparison.csv};
		\addplot[mark=*,mark size=1] table [col sep=comma,y index=2, x index=0] {data/post-op-pso-exafs-comparison.csv};

		\end{axis}
	\end{tikzpicture}
	\caption{OEC EXAFS Spectra Comparison}
	\label{fig:post-op-pso-best-exafs}
\end{figure*}