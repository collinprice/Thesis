\section{Problem Encoding}

A molecule consists of a number of atoms. Each of these atoms has its own 3-dimensional position within the molecule. For the structure refinement problem the individual 3-dimensional position values are not important. The important information about this problem is how the atoms are positioned with respect to each other.

\subsection{Representation}

\textbf{***TBA: Might change from list of 3-dimensional coordinates to linear list of numbers.***}

The chromosome representation consists of a list of 3-dimensional coordinates in space, where each position is assigned to a specific atom in the atomic structure. The actual position of each atom is not relevant because the goal is to determine the relative distances between the atoms. A subset of an individual can be seen in Table~\ref{table:sampleChromosome}. The units of measurement for each atom position are measured in Angstroms (\AA).

\begin{table}
\caption{Sample Chromosome Representation}
\label{table:sampleChromosome}
\centering
\normalsize
\begin{tabular}{ | l | l | l |}
  \hline
    X & Y & Z \\ \hline
    14.451 & -13.346 & 1.133 \\ \hline
    15.336 & -13.488 & 2.014 \\ \hline
    13.005 & -13.364 & 1.452 \\ \hline
    0.019 & 0.011 & 0.045 \\ \hline
    ... & ... & ... \\ \hline
\end{tabular}
\end{table}

The individuals for the PSO and DE had to be modified to better suit these algorithms. Figure \ref{fig:postOpExplain} demonstrates how the individuals were converted from a list of 3-dimensional positions to a single list of floating point values. In order to evaluate the fitness the list was translated back to a list of 3-dimensional positions.