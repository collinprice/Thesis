\documentclass[]{beamer}
% Class options include: notes, notesonly, handout, trans,
%                        hidesubsections, shadesubsections,
%                        inrow, blue, red, grey, brown

\usepackage{pgfplots}
\usepackage{algorithm}
\usepackage{algorithmic}
\usepackage{stfloats}

% Theme for beamer presentation.
\usepackage{beamerthemesplit} 
% Other themes include: beamerthemebars, beamerthemelined, 
%                       beamerthemetree, beamerthemetreebars  

\title{CIBCB Paper Outline?}    % Enter your title between curly braces
\author{Collin Price}                 % Enter your name between curly braces
\institute{Brock University}      % Enter your institute name between curly braces
\date{\today}                    % Enter the date or \today between curly braces

\begin{document}

% Creates title page of slide show using above information
\begin{frame}
  \titlepage
\end{frame}

% \section[Introduction]{}

% Introduction

% EXAFS
% IFEFFIT

\section{Overview of the Method}

\begin{frame}
  \frametitle{EXAFS}

  \begin{itemize}
    \item X-Ray Absorption Fine Structure (XAFS) is a method used to measure the absorption coefficient of a material as a function of energy.
    \item EXAFS can be use to identifying properties of a molecule but they do not provide enough detail.
    \item We can identify how far atoms are from each other but we do not know their dihedral angles.
    \item The energy spectra given off by the molecule is unique for its structure meaning that we can create our own atomic structure and obtain an EXAFS spectra and compare the results.
    \item IFEFFIT application is used to simulate an EXAFS experiment.
  \end{itemize}
\end{frame}

\begin{frame}
  \frametitle{Sample EXAFS Spectra}

  \begin{figure}
  \begin{center}
  \begin{tikzpicture}
    \begin{axis}[
      width=10cm,
      height=6cm,
      grid=both,
    ]

    \addplot[mark=x] table [col sep=comma,y index=1, x index=0] {results/best_run/for_chart.csv};

    \end{axis}
  \end{tikzpicture}
  \end{center}
  \end{figure}

\end{frame}

\section{The Evolutionary Algorithm}

\begin{frame}
  \frametitle{Baseline Atomic Structure}

  \begin{itemize}
    \item Initial atomic structure is put into a simulation.
    \item Atoms are allowed to move freely to minimize energy.
    \item Generates baseline atomic structure.
  \end{itemize}
\end{frame}

\begin{frame}
  \frametitle{Initial Population Generation}

  \begin{itemize}
    \item A second simulation is run with baseline atomic structure.
    \item Atoms are heated up to cause them to move rapidly.
    \item Atoms begin to oscillate.
    \item Snapshots(10000) of atomic strcture are recorded.
    \item Top(300) atomic structures are used.
  \end{itemize}
\end{frame}

\begin{frame}
  \frametitle{Chromosome Representation}

  \begin{itemize}
    \item List of 3-dimensional coordinates.
    \item The actual positions of each atom is not relevant because we are only concerned with the position each atom relative to each other.
  \end{itemize}

  \begin{table}[htp]
  \centering
  \begin{tabular}{ | l | l | l |}
    \hline
      X & Y & Z \\ \hline
      14.451 & -13.346 & 1.133 \\ \hline
      15.336 & -13.488 & 2.014 \\ \hline
      13.005 & -13.364 & 1.452 \\ \hline
      0.019 & 0.011 & 0.045 \\ \hline
      7.816 & -10.174 & 3.906 \\ \hline
      12.724 & -10.266 & 4.814 \\ \hline
      13.506 & -13.217 & 5.94 \\ \hline
      15.394 & -10.779 & 6.204 \\ \hline
  \end{tabular}
  \end{table}

\end{frame}

\begin{frame}
  \frametitle{Recentering}

  \begin{algorithmic}

  \IF{population has converged to minimum diversity}
    \STATE remove all duplicate individuals;
    \WHILE{population not full}
      \STATE insert random draw from generated individuals into population;
    \ENDWHILE
  \ENDIF

  \end{algorithmic}

\end{frame}

\begin{frame}
  \frametitle{Crossover}

  \begin{itemize}
    \item Single-point.
  \end{itemize}

  \begin{table}[htp]
  \begin{tabular}{ | l | l | l |}
    \hline
      X & Y & Z \\ \hline
      \bf14.451 & \bf-13.346 & \bf1.133 \\ \hline
      \bf15.336 & \bf-13.488 & \bf2.014 \\ \hline
      \bf13.005 & \bf-13.364 & \bf1.452 \\ \hline
      0.019 & 0.011 & 0.045 \\ \hline
      7.816 & -10.174 & 3.906 \\ \hline
      12.724 & -10.266 & 4.814 \\ \hline
      13.506 & -13.217 & 5.94 \\ \hline
      15.394 & -10.779 & 6.204 \\ \hline
  \end{tabular}
  \begin{tabular}{ | l | l | l |}
    \hline
      X & Y & Z \\ \hline
      14.451 & -13.346 & 1.133 \\ \hline
      15.336 & -13.488 & 2.014 \\ \hline
      13.005 & -13.364 & 1.452 \\ \hline
      \bf0.019 & \bf0.011 & \bf0.045 \\ \hline
      \bf7.816 & \bf-10.174 & \bf3.906 \\ \hline
      \bf12.724 & \bf-10.266 & \bf4.814 \\ \hline
      \bf13.506 & \bf-13.217 & \bf5.94 \\ \hline
      \bf15.394 & \bf-10.779 & \bf6.204 \\ \hline
  \end{tabular}
  \end{table}

\end{frame}

\begin{frame}
  \frametitle{Mutation}

  \begin{itemize}
    \item Single point from chromosome is selected.
    \item Point is moved randomly by 0.05.
    \item 0.05 was selected after determining that it was the minimum move required to make a difference in the EXAFS spectra.
  \end{itemize}

\end{frame}

\begin{frame}
  \frametitle{Selection}

  \begin{itemize}
    \item 3-tournament selection.
  \end{itemize}

\end{frame}

\begin{frame}
  \frametitle{Fitness}

  \begin{itemize}
    \item root-mean-square deviations(RMSD) between the two spectras.
  \end{itemize}

  \begin{figure}
  \begin{center}
  \begin{tikzpicture}
    \begin{axis}[
      width=10cm,
      height=6cm,
      grid=both,
      legend entries={Experimental,Calculated}
      % xlabel={Generations},
      % ylabel={RMSD}
    ]

    \addplot[mark=x] table [col sep=comma,y index=1, x index=0] {results/best_run/for_chart.csv};
    \addplot[mark=*] table [col sep=comma,y index=2, x index=0] {results/best_run/for_chart.csv};

    \end{axis}
  \end{tikzpicture}
  \end{center}
  \end{figure}

\end{frame}

\begin{frame}
  \frametitle{Best GA Parameters}

  \begin{table}[htp]
  \caption{GA Parameters}
  \centering
  \begin{tabular}{ l r }
    \hline
      Population size & 50 \\
      Crossover rate & 0.8 \\
      Mutation rate & 0.2 \\
      Elitism & True \\
      Number of recentering attempts & 5 \\
      Max convergence percentage before recentering & 5\% \\
    \hline
  \end{tabular}
  \end{table}

\end{frame}

\begin{frame}
  \frametitle{Full GA Parameters}


  \begin{table}[t]
  \tiny
  \begin{tabular}{ | l | l | l | l | l | l | l | l | l | l | }
    \hline
    Name & Runs & Pop. & Gen. & Crossover & Mutation & Elitism & Conv. Rate & Recentering &  Avg. Best \\ \hline \hline
    Test1 & 10 & 50 & 30 & 80\% & 20\% & False & - & - & 325.55 \\ \hline
    Test4 & 10 & 50 & 30 & 80\% & 10\% & False & - & - & 342.04 \\ \hline
    Test5 & 10 & 50 & 30 & 70\% & 30\% & False & - & - & 294.56 \\ \hline
    Test2 & 10 & 50 & 30 & 80\% & 10\% & True & - & - & 319.14 \\ \hline
    Test3 & 10 & 50 & 30 & 70\% & 30\% & True & - & - & 286.98 \\ \hline
    Test6 & 10 & 50 & 30 & 80\% & 20\% & True & - & - & 320.17 \\ \hline
    Test7 & 10 & 50 & 67 & 80\% & 20\% & True & 10\% & 3 & 271.49 \\ \hline
    Test8 & 10 & 50 & 81 & 80\% & 20\% & True & 5\% & 3 & 265.76 \\ \hline
    Test9 & 10 & 50 & 107 & 80\% & 20\% & True & 10\% & 5 & 263.36 \\ \hline
    Test10 & 10 & 50 & 124 & 80\% & 20\% & True & 5\% & 5 & 267.47 \\ \hline
    Test11 & 10 & 50 & 67 & 70\% & 30\% & True & 5\% & 5 & \textbf{260.56} \\ \hline
    Test12 & 10 & 50 & 81 & 70\% & 30\% & True & 10\% & 5 & 266.68 \\ \hline
  \end{tabular}
  \end{table}

\end{frame}

\begin{frame}
  \frametitle{Best Run}

  \begin{figure}
  \begin{center}
  \begin{tikzpicture}
    \begin{axis}[
      width=10cm,
      height=6cm,
      grid=both,
      title={Best Run},
      legend entries={Best,Average},
      xlabel={Generations},
      ylabel={RMSD}
    ]

    \addplot[mark=x] table [col sep=comma,y index=1, x index=0] {results/best_run/results.csv};
    \addplot[mark=*] table [col sep=comma,y index=2, x index=0] {results/best_run/results.csv};

    \end{axis}
  \end{tikzpicture}
  \end{center}
  \end{figure}

\end{frame}

\begin{frame}
  \frametitle{Further Experiments}

  \begin{itemize}
    \item Subsets of atoms. Currently using Ma, Ca, O, C, N, H.
    \item Reduce x-range in EXAFS spectra RMSD.
    \item CILIB - PSO
    \item Use PSO for local search at end?
  \end{itemize}

\end{frame}

% \section[Results]{}

% \section[Conclusion and Future Work]{}

% Recentering
% Crossover
% Mutation
% Selection
% Fitness
% GA Parameters
% Comparison
% Analysis
% Conclusion
% Future Work

% \section{Simple slide with three points shown all at once}

% \begin{frame}
%   \frametitle{Simple slide with three points shown all at once}   % Insert frame title between curly braces

%   \begin{itemize}
%   \item Point 1
%   \item Point 2
%   \item Point 3
%   \end{itemize}
% \end{frame}
% \note[enumerate]       % Add notes to yourself that will be displayed when
% {                      % typeset with the notes or notesonly class options
% \item Note for Point 1   
% \item Note for Point 2   
% }

% \subsection{Simple slide with three points shown in succession}

% \begin{frame}
%   \frametitle{Simple slide with three points shown in succession}   % Insert frame title between curly braces

%   \begin{itemize}
%   \item<1-> Point 1 (Click ``Next Page'' to see Point 2) % Use Next Page to go to Point 2
%   \item<2-> Point 2  % Use Next Page to go to Point 3
%   \item<3-> Point 3
%   \end{itemize}
% \end{frame}
% \note{Speak clearly}  % Add notes to yourself that will be displayed when
%                       % typeset with the notes or notesonly class options


% \section{Slide with two columns: items and a graphic}

% \begin{frame}
%   \frametitle{Slide with two columns: items and a graphic}   % Insert frame title between curly braces
%   \begin{columns}[c]
%   \column{2in}  % slides are 3in high by 5in wide
%   \begin{itemize}
%   \item<1-> First item
%   \item<2-> Second item
%   \item<3-> ...
%   \end{itemize}
%   \column{2in}
%   \framebox{Insert graphic here % e.g. \includegraphics[height=2.65in]{graphic}
%   }
%   \end{columns}
% \end{frame}
% \note{The end}       % Add notes to yourself that will be displayed when
		     % typeset with the notes or notesonly class options

\end{document}
